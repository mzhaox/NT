\documentclass[11pt]{amsart}
\usepackage{amsmath, amsthm, amssymb, hyperref}
\usepackage{graphicx}
\usepackage[square]{natbib}
\usepackage{hyperref}
\usepackage{fullpage}
\usepackage{mathtools}
\usepackage{enumitem}

\newtheorem*{theorem}{Theorem}
\newtheoremstyle{named}{}{}{\itshape}{}{\bfseries}{.}{.5em}{\thmnote{#3's }#1}
\theoremstyle{named}
\newtheorem*{namedtheorem}{Theorem}
\newtheorem{definition}{Definition}
\newtheorem{prop}{Proposition}
\newtheorem{lem}{Lemma}

\newcommand{\ie}{{\it i.e. }}
\newcommand{\com}[1]{\color{blue}{ #1 }\color{black}}
\newcommand{\cl}[1]{{\mathrm{Cl}(#1)}}
\newcommand{\roi}[1]{\mathcal{O}_{#1}}
\newcommand{\gal}[3]{\mathrm{Gal}(#1 #2 #3)}
\newcommand{\iso}{\cong}
\newcommand{\rats}{\mathbb{Q}}
\newcommand{\reals}{\mathbb{R}}
\newcommand{\cmplx}{\mathbb{C}}
\renewcommand{\prime}{\mathfrak{p}}
\newcommand{\oprime}{\mathfrak{P}}
\newcommand{\oprimealt}{\oprime^o}
\newcommand{\frob}[1]{\mathrm{Fr}_{#1}}
%\newcommand{\legndr}[2]{\left(\dfrac{#1}{#2}\right)}
\newcommand{\legndr}[2]{\genfrac{(}{)}{}{}{#1}{#2}}
\newcommand{\N}[1]{N(#1)}
\newcommand{\Zmod}[1]{\mathbb{Z}/#1 \mathbb{Z}}
\newcommand{\Z}{\mathbb{Z}}
\newcommand{\Zn}{\mathbb{Z}/n\mathbb{Z}}
\newcommand{\Znu}{\left(\mathbb{Z}/n\mathbb{Z}\right)^\times}
\newcommand{\Zp}{\mathbb{Z}/p\mathbb{Z}}
\newcommand{\Zpu}{\left(\mathbb{Z}/p\mathbb{Z}\right)^\times}
\newcommand{\nnreals}{\mathbb{R}^{\geq 0}}
\newcommand{\Q}{\mathbb{Q}}
\newcommand{\Qp}{\Q_p}
\newcommand{\tr}{\mathrm{tr}}
\newcommand{\sidele}{\mathbb{A}_K^{\times S}}
\newcommand{\sadele}{\mathbb{A}_K^{S}}
\newcommand{\idele}{\mathbb{A}_K^{\times}}
\newcommand{\adele}{\mathbb{A}_K}

\begin{document}

\section{Grunwald-Wang Counterexamples}

\subsection{For $\Q$} We show that $16$ is an $8$-th power in $\Q_v$ for
$v \neq 2$. To start, we have
\[ X^8 - 16 = (X^2 - 2 )(X^2 + 2)(X^2 - 2X + 2)(X^2 + 2X + 2). \]
The roots of $X^2 \pm 2X + 2$ are $1 \pm \sqrt{-1}, -1 \pm \sqrt{-1}$,
respectively. Thus, the splitting field of $X^8 - 16$ is
$K := \Q(\sqrt{2}, \sqrt{-1})$. Hence for $p$ odd, we need to establish that one
of $\sqrt{\pm 2}, \sqrt{-1}$ are in $K$. We do so by looking for integral
solutions  to $X^8 - 16$, in order to apply Hensel's lemma. If $2, -1$ are not
squares in $\mathbb{F}_p$, then multiplicativity of the Legendre symbol shows
$-2$ is a square. If $2, -2$ aren't, then $-4$ is, and since $p$ is odd, this
means $-1$ is. If $-1, -2$ aren't, then $2$ is. This means a modulo $p$ solution
to $X^8 - 16$ is guaranteed, and since none of these solutions are zero, by
Hensel's lemma there is a root to $X^8 - 16$ is $\Z_p$, for odd $p$.

\subsection{For $\mathbb{Q}[\sqrt{7}]$} For $p$ an odd prime, the same use of
Hensel's lemma above shows that
$\Q_p[\sqrt{7}] = \Q_p[\sqrt{2}, \sqrt{-1}][\sqrt{7}].$ For $p = 2$,
$\Q_2[\sqrt{7}] = \Q_2[\sqrt{2}]$. Let $\alpha = \sqrt{7}$. Then
\begin{align*}
  \alpha^2 - 2\alpha + 4 + 2\alpha - 4  - 8 & = -1 \\
  \Rightarrow (\alpha -2)^2 + 2(\alpha - 2) - 7 & = 0
\end{align*}
Then $\alpha - 2 = -1 \pm 4\sqrt{2}$, so $(\alpha - 1)/4 = \pm \sqrt{2}$. Thus
$X^8 - 16$ has a root in $\Q_2(\sqrt{7})$.

\subsection{Relation to Grunwald-Wang} 


\section{Norms are Local Norms}

\subsection{Finite, Cyclic Extensions} Let $L/K$ be finite cyclic extensions of
number fields. We will show $a \in N_{L/K} L$ if and only if it is in
$N_{L_w/K_v} L_w$ for all places $w$ of $L$. This is true since we have the
following maps
\begin{align*}
  K^\times / N_{L/K} L^\times \cong H^2(G(L/K), L^\times) & \xhookrightarrow{}
H^2(G(L/K), \mathbb{A}^\times_L) \\
  & \cong \bigoplus_v H^2(G(L_w/K_v), L_w^\times)
  \cong\bigoplus_v K_v^\times /  N_{L_w/K_v} L_w^\times.
\end{align*}

\subsection{Counterexample for Non-cyclic Extensions} Let
$L = \Q(\sqrt{13}, \sqrt{17})$. We show that $25$ is not a global norm but it
is everywhere a local norm. Let
$\alpha = a + b\sqrt{13} + c\sqrt{17} + d\sqrt{17 \cdot 13}$. Let
$x = a + b\sqrt{13}, y = c + d\sqrt{13}$. Suppose $25 = N(\alpha)$. Then
\begin{align*}
  25 & = N_{\Q(\sqrt{13}/\Q}(N_{L/\Q(\sqrt{13})} \alpha) \\
     & = N_{\Q(\sqrt{13})/\Q}(x^2 - 17y^2).
\end{align*}

\section{Hilbert Class Field}

\subsection{Hilbert Class Field} Let
$U = K^\times \prod_{v | \infty} K_v^\times \prod_{v \nmid \infty}
\roi{K_v}^\times.$ We claim this is the subgroup of $C_K$ corresponding
to the Hilbert class field. By the existence theorem, there is an $H/K$ such
that $N_{H/K} C_H = U$. Then by Artin reciprocity,
\[ C_K/U \xrightarrow{\sim} \mathrm{Gal}(H/K) .\]
For $v$ an infinite place, looking at the local reciprocity map gives
$K_v^\times/K_v^\times \xrightarrow{\sim} \mathrm{Gal}(H_w/K_v),$ so $v$ is
split completely. For $v$ a finite place,
\[\roi{K_v}^\times/\roi{K_v}^\times K^\times
\subset K_v^\times/\roi{K_v}^\times K^\times\] is trivial, and so the local
reciprocity map sends it to the identity element of $\mathrm{Gal}(H/K)$. Thus,
$v$ is unramified.

For the narrow Hilbert class field, for $v | \infty$ we take $(K_v^\times)^2$.
Then the $\mathrm{Gal}(H_w/K_v)$, for $w | v$, have order two or one depending
on if $K_v$ is real or complex.

%Finally, we have
%\[ C_K/U = \left(\prod_{v \nmid \infty} K_v^\times/\roi{K_v}^\times\right)
%\left/K^\times.\]
%Since $K_v^\times/\roi{K_v}^\times \cong \Z,$ and $K^\times$ is isomorphic to
%the principal fractional ideals, this is isomorphic to the class group
%$\mathrm{Cl}(K)$.

\end{document}

