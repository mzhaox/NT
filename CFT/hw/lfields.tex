\documentclass[12pt]{amsart}
\usepackage{amsmath, amsthm, amssymb, hyperref}
\usepackage{graphicx}
\usepackage[square]{natbib}
\usepackage{hyperref}
\usepackage{fullpage}
\usepackage{mathtools}

\newtheorem*{theorem}{Theorem}
\newtheoremstyle{named}{}{}{\itshape}{}{\bfseries}{.}{.5em}{\thmnote{#3's }#1}
\theoremstyle{named}
\newtheorem*{namedtheorem}{Theorem}
\newtheorem{definition}{Definition}
\newtheorem{prop}{Proposition}
\newtheorem{lem}{Lemma}

\newcommand{\ie}{{\it i.e. }}
\newcommand{\com}[1]{\color{blue}{ #1 }\color{black}}
\newcommand{\cl}[1]{{\mathrm{Cl}(#1)}}
\newcommand{\roi}[1]{\mathcal{O}_{#1}}
\newcommand{\gal}[3]{\mathrm{Gal}(#1 #2 #3)}
\newcommand{\iso}{\cong}
\newcommand{\rats}{\mathbb{Q}}
\newcommand{\reals}{\mathbb{R}}
\newcommand{\cmplx}{\mathbb{C}}
\newcommand{\prim}{\mathfrak{p}}
\newcommand{\oprime}{\mathfrak{P}}
\newcommand{\oprimealt}{\oprime^o}
\newcommand{\frob}[1]{\mathrm{Fr}_{#1}}
\newcommand{\legndr}[2]{\genfrac{(}{)}{}{}{#1}{#2}}
\newcommand{\N}[1]{N(#1)}
\newcommand{\Zmod}[1]{\mathbb{Z}/#1 \mathbb{Z}}
\newcommand{\Z}{\mathbb{Z}}
\newcommand{\Zn}{\mathbb{Z}/n\mathbb{Z}}
\newcommand{\Znu}{\left(\mathbb{Z}/n\mathbb{Z}\right)^\times}
\newcommand{\Zp}{\mathbb{Z}/p\mathbb{Z}}
\newcommand{\Zpu}{\left(\mathbb{Z}/p\mathbb{Z}\right)^\times}
\newcommand{\nnreals}{\mathbb{R}^{\geq 0}}
\newcommand{\Q}{\mathbb{Q}}
\newcommand{\Qp}{\Q_p}
\newcommand{\tr}{\mathrm{tr}}

\begin{document}
\section{Solutions to $x^2 - 7$ in $\Qp$} We first determine
when $7$ is a square mod $p$. By quadratic reciprocity, we have \[ \legndr 7 p \legndr p 7 =
(-1)^{\frac{p - 1}{2}}.\] The second Legendre symbol $\legndr p 7$ is 1 when $p$ is a quadratic
residue mod 7, i.e. $1, 2, 4$.
\begin{itemize}
\item When $p \equiv 1 \pmod 7$, we see that such primes must be of the form
$p = 28k + 1, 28k + 15$.
\item When $p \equiv 2 \pmod 7$, we have to consider when $p$ is of the form
$p = 28k + 9, 28k + 23$.
\item When $p \equiv 4 \pmod 7$, we have to consider when $p$ is of the form
$p = 28k + 25, 28k + 11$.
\end{itemize}
In all of these cases, the right hand side is $1$ only in the first case, i.e. when
$p \equiv 1, 9, 25 \pmod {28}$. When $p \equiv 3, 5, 6 \pmod 7$. we have the following cases to
consider.
\begin{itemize}
\item When $p \equiv 3$, $p = 28k + 3, 28k + 17$.
\item When $p \equiv 5$, $p = 28k + 19, 28k + 5$.
\item When $p \equiv 6$, $p = 28k + 27, 28k + 13$.
\end{itemize} In all of these cases, the right hand side is
$-1$ only in the first case, i.e. when $p \equiv 3, 19, 27 \pmod{28}$. So when $p \equiv 1, 3, 9,
19, 25, 27$, we have an initial solution to start with Hensel's lemma. Otherwise, there can be no
solution.

Assume $p \neq 2, 7$. Suppose we have a solution $x_0$ to $f(x) = x^2 - 7 \equiv 0 \pmod p$. Then
$f'(x_0) = 2x_0 \not\equiv 0 \pmod p$. By Hensel's lemma, we are done. On the other hand, if we have
an $x \in \Q_2$ with $x^2 - 7$, and we reduce modulo $4$, then we arrive at a contradiction, since
the only quadratic residues modulo $4$ are $0, 1$ and $7 \equiv 3 \pmod 4$. And of course, there can
be no solution in $\Q_7$ since that would mean $7$ is not prime in $\Q_7$.

Hence, when $p \equiv 1, 3, 9, 19, 25, 27 \pmod{28}$, $x^2 - 7$ has a root in $\Qp$.

\section{The different}

\subsection{Different of a power basis} Let the conjugates of $\beta$ be $\beta_i$ for $i =
1,..,n$. Then let $f(X) = \prod_i (X - \beta_i)$ and let $f_i(X) = \prod_{j, j \neq i} (X -
\beta_j)$. Let $\pi_i = \prod_{j, j \neq i} (\beta_i - \beta_j)$. Then we claim that $1 = \sum_j
f_j(X)/\pi_j$. The sum is a polynomial of degree $n - 1$, so we just need to check equality at $n$
places. Notice that $f_j(\beta_i) = \delta_{ij}$, hence the equality holds for $X = \beta_i,$ $i =
1,...,n$. Noticing that $f'(\beta_j) = \pi_j,$ multiplying each summand by $1 = (X - \beta_j)/(X -
\beta_j)$, and dividing by $f(X)$ gives us \[ \dfrac{1}{f(X)} = \sum_{k = 1}^n \dfrac{1}{f'(\beta_k)
(X - \beta_k)}. \] To establish the second equality, notice that we can expand \[ \dfrac{1}{X -
\beta_k} = \dfrac{1}{X} \dfrac{1}{1 - \frac{\beta_k}{X}}, \] as a geometric series. We can commute
the sums to obtain the second equality, \[\dfrac{1}{f(X)} = \sum_{i = 1}^\infty X^{-i} \tr_{L/K}
\dfrac{\beta^{i - 1}}{f'(\beta)}.\] To actually compute these traces, we can write \[
\dfrac{1}{f(X)} = \dfrac{1}{X^n(1 - a(1/X))} = \dfrac{1}{X^n} (1 + a(1/X) + a(1/X)^2 + \cdots ), \]
where $a(1/X)$ is a polynomial in $1/X$ with no constant term. By comparing coefficients, we see
that we must have $\tr_{L/K} \frac{\beta^{i - 1}}{f'(\beta)}$ equal to $0$ for $i = 1, ..., n - 1$,
and $1$ for $i = n$. Moreover, since $f$ had integral coefficients, so will $a(1/X)$, so for $i >
n$, the traces are integral.

Now, write $x f'(\beta) = \sum_{i = 0}^{n - 1} a_i \beta^i$, so
$x = \sum_{i = 0}^{n - 1} a_i \frac{\beta^i}{f'(\beta)}.$ Then we have
\begin{align*}
\tr_{L/K} (x \beta^j) & = \sum_{i = 0}^{n - 1} a_i \tr_{L/K}{\beta^{i + j}/f'(\beta)} \\
& = a_{n - j} + \sum_{i = n - j + 1}^{n + j - 1} \tr_{L/K}{\beta^{i}/f'(\beta)}.
\end{align*}
Thus, by sequentially setting $j = 1, ..., n$, we can verify that $a_{n - 1}, ..., a_0$ are
integral, respectively.

Thus, we can conclude that the $\beta^i/f'(\beta)$ form an integral basis for the inverse different. 
Then by noting that the inverse different is therefore equal to $\frac{1}{f'(\beta)} \roi L,$
the different must be $f'(\beta) \roi L$.

\subsection{Factorization of the different}
\subsubsection{Complete Approach} Let $S$ be a multiplicative subset of $\mathcal{O}_L$.
Then we claim that
\[ \mathcal{D}_{S^{-1} \mathcal{O}_L/S^{-1} \mathcal{O}_K} = S^{-1} \mathcal{D}_{\mathcal{O}_L/\mathcal{O}_K}. \]

Now if $x \in D_{L_w/K_v}$ for all $w | v$, then 
\[\tr_{L_w/K_v}(x \mathcal{O}_L) \subset \tr_{L_w/K_v}(x\mathcal{O}_{L_w}) \subset \mathcal{O}_{K_v}.\]
Then since $\tr_{L/K} (x \mathcal{O}_L) \subset K$, we actually have
$\tr_{L/K}(x \mathcal{O}_L) \subset \mathcal{O}_K.$

Let $x \in D_{L/K}$. Say $w$ is the valuation which has not been killed by localization, $w' | v$
have been killed. For $y \in \mathcal{O}_{L_w}$, take $\hat{y} \in \mathcal{O}_L$ approximating $y$
and approximating $0$ for other $w' | v$. Then
\[ \mathrm{tr}_{L/K}(x \hat{y}) = \tr_{L_w/K_v}(x \hat{y}) + \sum_{w' | v} \tr_{L_w'/K_v}(x \hat{y}),\]
Since the LHS is in $\mathcal{O}_K$ and the terms of the sum on the right are in $\mathcal{O}_{K_v}$,
we must have $\tr_{L_w/K_v}(x \hat{y})$ also in $\mathcal{O}_{K_v}$. Since $\hat{y}$ approximates
$y$, $\tr_{L_w/K_v}(xy) \in \mathcal{O}_{K_v}$.

\subsubsection{Incomplete Approach} The comments on the homework I turned in said to look at the isomorphism
\[ \mathcal{O}_L \otimes_{\mathcal{O}_K} \mathcal{O}_{K_v} \cong \prod_{w|v} \mathcal{O}_{L_w}. \]
Motivated by the isomorphism $\mathcal{D}_{L/K}^{-1} \cong \mathrm{Hom}_{\mathcal{O}_K}(\mathcal{O}_L, \mathcal{O}_K),$
we can take homs of both sides:
\[ \mathrm{Hom}_{\mathcal{O}_{K_v}}(\mathcal{O}_L \otimes_{\mathcal{O}_K} \mathcal{O}_{K_v}, \mathcal{O}_{K_v})
\cong \prod_{w|v} \mathrm{Hom}_{\mathcal{O}_{K_v}}(\mathcal{O}_{L_w}, \mathcal{O}_{K_v})
\cong \prod_{w|v} \mathcal{D}_{L_w/K_v}^{-1}.\]
But the left hand side is isomorphic to $\mathrm{Hom}_{\mathcal{O}_K}(\mathcal{O}_L, \mathrm{Hom}_{\mathcal{O}_{K_v}}(\mathcal{O}_{K_v}, \mathcal{O}_{K_v}))
\cong \mathrm{Hom}_{\mathcal{O}_K}(\mathcal{O}_L, \mathcal{O}_{K_v})$. A subset of this is
$\mathcal{D}_{L/K}^{-1} \cong \mathrm{Hom}_{\mathcal{O}_K}(\mathcal{O}_L, \mathcal{O}_K)$, but this
is a direct product as opposed to an ideal product.

\subsection{Valuations of the Different} \label{vals}
We can compute
\[v_L(f'(\beta)) = v_L\left(\prod_{\substack{\gamma \neq \mathrm{id} \\
\gamma \in G(L/K)}} (\beta - \gamma \beta)\right) = \sum_{\substack{\gamma \neq \mathrm{id} \\
\gamma \in G(L/K)}} v_L(\beta - \gamma \beta).
\]
Notice that $v_L(\beta - \gamma \beta)$ counts the number of lower ramification groups $\gamma$
is an element of. Thus, if $G_i$ are the lower ramification groups,
\[ v_L(f'(\beta)) = \sum_i |G_i| - 1. \]

\section{Prime power cyclotomic field}
\subsection{Units and valuations} Let $i
\in \left(\Zmod {p^n}\right)^\times$. Then $\zeta^i$ and $\zeta$ are primitive $p^n$-th roots of
unity. Let $j$ be such that $ij \equiv 1 \pmod{p^n}$. Certainly, we have $\frac{1 - \zeta^i}{1 -
\zeta} \in \roi K$. We also have \[\frac{1 - (\zeta^i)^j}{1 - \zeta^i} = 1 + \zeta^i + \zeta^{2i} +
\cdots + \zeta^{(j - 1)i} \in \roi K.\] However, the left hand side is $\frac{1 - \zeta}{1 -
\zeta^i} \in \roi K$. Hence for
$i \in \left(\Zmod {p^n}\right)^\times$, $\frac{1 - \zeta^i}{1 - \zeta}$ is a unit.

By evaluating the $p^n$ cyclotomic polynomial \[ \prod_{k, (k, p) = 1} (X - \zeta^k) =
\dfrac{X^{p^n} - 1}{X^{p^{n - 1}} - 1} = 1 + X^{p^{n - 1}} + \cdots + X^{(p - 1) p^{n - 1}},\] at $X
= 1$, we find that \[ p = \prod_{k, (k, p) = 1} (1 - \zeta^k) = \prod_{k, (k, p) = 1} (1 - \zeta)
\dfrac{1 - \zeta^k}{1 - \zeta}. \] Taking valuations, we find $v_K(p) = \varphi(p) v_K(1 - \zeta)$.

\subsection{Isomorphism and uniformizer} If the $p^n$ cyclotomic polynomial is irreducible, then
it has order $\varphi(p^n)$. Thus the orders of $G(K/\Qp)$ and $\left(\Z/p^n \Z \right)^\times$ are
the same, and $\kappa$ is an isomorphism.

Irreducibility can be verified by Eisenstein's criterion. Let $\phi(X)$ be the $p^n$ cyclotomic.
Then notice that $\phi(X + 1)$ has leading coefficient 1, and has constant term divisible by $p$ but
not $p^2$. To show that every other term of $\phi(X + 1)$ is divisible by $p$, notice that modulo
$p$,
\begin{align*}
\phi(X + 1) & = \dfrac{(X + 1)^{p^n} - 1}{(X + 1)^{p^{n - 1}} - 1} \\
& = \dfrac{X^{p^n}}{X^{p^{n - 1}}} = X^{\varphi(p^n)}.
\end{align*}
Thus, every term other than the leading term is divisible by $p$. Thus, by Eisenstein and Gauss,
$\phi(X)$ is irreducible over $K[X]$.

This allows us to conclude the following inequalities:
\begin{align*}
\varphi(p^n) = [K : \Qp] & \geq e_{K|\Qp} \\
e_{K|\Qp} = v_K(p) = v_K(1 - \zeta) \varphi(p) & \geq \varphi(p^n).
\end{align*}
Thus $p$ is totally ramified, and $v_K(1 - \zeta) = 1$, so $1 - \zeta$ is a uniformizer.

\subsection{Lower Ramification Groups} Locally, we know that $\mathcal{O}_K = \Z_p [\zeta]$ so we
just need to check the action of Galois automorphisms on $\zeta$. For $i = -1$, the condition
$v_K(\sigma(\zeta) - \zeta) \geq i + 1$ is automatically true, hence $G(L/K)_{-1} = G(L/K)$. For $i
= 0$, for any Galois automorphism we have $v_K(\sigma(\zeta) - \zeta) = v_K(\zeta(1 - \zeta)(1 +
\cdots)) \geq v_K(1-\zeta) = 1.$

For $i > 0$, suppose $\sigma(\zeta) = \zeta^j$ or $j := \kappa(\sigma)$. Then \[v_K(\sigma(\zeta) -
\zeta) = v_K(\zeta(\zeta^{j - 1} - 1)) = v_K(\zeta^{j - 1} - 1),\] and if we let $v := v_p(j - 1)$,
then $\zeta^{j - 1}$ is a primitive $p^{n - v}$ root of unity and $\zeta^{j - 1} - 1$ is a
uniformizer. To compute $v_K(\zeta^{j - 1} - 1)$, we can use transitivity of ramification indices.
Thus, $v_K(\zeta^{j - 1} - 1) = p^v$.
Then for $p^{k - 1} \leq i < p^k$, $\sigma \in G_i$ if and only if
$p^v \geq i + 1 > p^{k - 1}$, so $v \geq k$. This also means $j \equiv 1 \pmod{p^k}$. Thus, $\sigma$
fixes $p^k$-th roots of unity. Hence $G_i = G(K/\Qp(\zeta^{p^{n - k}})).$

\subsection{Different} Combining the formula from \ref{vals} and our work above, the different is
given by $(1 - \zeta)^a$, where
\begin{align*}
a & = \varphi(p^n) + \sum_{i = 1}^{n - 1} p^{i - 1}(p - 1) \varphi(p^n)/\varphi(p^i) \\
& = \varphi(p^n) + \sum_{i = 1}^{n - 1} p^{n - 1}(p - 1) = n p^{n - 1} (p - 1).
\end{align*}

\section{Computations for a Biquadratic Field}
Let $K = \Q_2[\sqrt{-1}, \sqrt{2}]$, and $K' = \Q_2[\sqrt{-1}]$. Let $i := \sqrt{-1}$. Then $1 + i$,
by our work above, is a uniformizer of $K'$. If we find a uniformizer for $K/K'$, we will therefore
have a uniformizer for $K/\Q_2$. Observe that
\[ 1 + i = (\sqrt{2} - 1)\left(1 + \dfrac{\sqrt{2}}{2} + \dfrac{i \sqrt{2}}{2}\right)^2, \] and that
$\sqrt{2} - 1$ is a unit, with inverse $1 + \sqrt{2}$. Let
$\alpha, \beta, \omega := \sqrt{2}/2, i \alpha, 1 + \alpha + \beta$. Then the Galois group is
generated by $\sigma_1, \sigma_2$ sending $\sqrt{2} \rightarrow -\sqrt{2}$ and $i \rightarrow -i$. 
We have the following:
\begin{itemize}
\item $\sigma_1(\omega) - \omega = - 2 \alpha (1 + i)$, which has valuation 4.
\item $\sigma_2(\omega) - \omega = - 2 \beta$, which has valuation 2.
\item $(\sigma_1 \circ \sigma_2)(\omega) - \omega = -2\alpha$, which has valuation 2.
\end{itemize}
Thus, the lower ramification groups are
\begin{enumerate}
\item $G_0 = G(K/\Q_2).$
\item $G_1 = G(K/\Q_2).$
\item $G_2 = G(K/\Q_2(i\sqrt{2})).$
\item $G_3 = G_2.$
\item $G_i = 0$, $i \geq 4$.
\end{enumerate}
\section{Eisenstein Polynomials}
\subsection{Eisenstein Polynomials Yield Totally Ramified Extensions}
Let $f$ be Eisenstein. We
first show it is irreducible. Suppose it factors as $f = gh$. Then modulo $p$,
$X^{\deg f} = \overline{g} \overline{h}$. However, $\mathbb{F}_p[X]$ is a principal ideal domain, hence a
UFD. So $p$ divides all but the leading coefficients of $g$ and $h$. But then $p^2$ divides
$f(0) = g(0) h(0)$, so we have a contradiction.

Now suppose $L = K[x]/(f(x)) \cong K[\alpha]$ is an extension of $K$ with $f$ Eisenstein. If $L$ is
separable, then $v_L$ uniquely extends $v_K$ and we have the following relation. Let $p$ be
a uniformizer of $K$, then we have \[1 = v_K(up) = v_K(N_{L/K}(\alpha)) = [L : K] v_L(\alpha),\] for 
some unit $u$. Thus, $L/K$ is totally ramified.

\subsection{Every Totally Ramified Comes From Eisenstein} Suppose the valuation $v_L$ is normalized.
Since $L/K$ is totally ramified, if $f = a_n X^n + \cdots + a_0$ is the minimal polynomial of a
uniformizer $\omega$ of $L$, $v_L(a_i) \equiv 0 \pmod n$. Consider the $v_L(a_i \omega^i)$.
Notice that \[v_L(a_i \omega^i) = v_L(a_i) + i \equiv i \pmod n.\] This means, among
$i = 0, ..., n - 1$, no two valuations are the same. Hence,
\[n = v_L(-\omega^n) = v_L(\sum_{i = 0}^{n - 1} a_i \omega^i) = min_i (i + v_L(a_i)).\]
Thus, $v_L(a_i) \geq n - i$ for every $i$. Combined with the fact that $v_L(a_i) \equiv 0 \pmod n$,
this shows $f$ is Eisenstein, since also $v_L(a_0) = v_L(\omega^n) = n$.
\end{document}
